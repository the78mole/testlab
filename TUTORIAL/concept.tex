\part{Konzept}

\section{Grundlegende �berlegungen}

Die gew�nschte Quelle und Senke erfordern eine Anpassung der Schaltung sowohl im Ein-, als auch im Ausgangspfad. Ist eine Leistungsverst�rkung am Ausgang gew�nscht, kommt ein einfacher NF-Kleinleistungsverst�rker zum Einsatz, die bereits in vielen Baus�tzen als Referenzdesign vorliegen. Schwieriger wird allrdings die Verf�lschung der digitalen Daten in Echtzeit. Dazu soll ein FPGA zum Einsatz kommen, der anhand mehrerer Module die gew�nschte Ver�nderung durchf�hrt. M�gliche Varianten der Implementierung sind folgende:

\begin{itemize}
	\item Look-Up Tables
	\begin{itemize}
		\item Mit Interpolation
		\item Ohne Interpolation\footnote{Kommt aufgrund ungen�gender Ressourcen nicht zum Einsatz}
	\end{itemize}
	\item Einfache Arithmetik
	\begin{itemize}
		\item Addition
		\item Multiplikation
		\item Potenzen niedriger Ordnung
	\end{itemize}
	\item Komplexe Berechnungen (CORDIC)
	\item Pseudo-Random Noise
\end{itemize}

