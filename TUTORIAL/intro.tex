\part{Einleitung}

\section{Motivation}

Im Rahmen eines Praktikums soll den Studierenden anhand eines Analog-Digital-Umsetzers das praktische Wissen zum Test integrierten Schaltungen vermittelt werden. Da die ausgelieferten AD-Wandler bereits getestet sind und nur mit geringen Abweichungen der Kennlinie zu rechnen ist, muss eine Schaltung entwickelt werden, welche die digitalen Werte so verf�lscht, dass diese einem fehlerbehafteten AD-Wandler �hnlich sind. Folgende Fehler sollen dabei modelliert werden:

\begin{itemize}
	\item Kennlinienfehler
	\item (starkes) Rauschen
	\item Quantisierungsfehler ("`missing Codes"', falsche Codes)
\end{itemize}

Die eingef�gten Fehler sollen in Ihrer Wirkung einstellbar sein. Weiterhin war gew�nscht, eine Audio-Signalquelle und einen Lautsprecher anschlie�en zu k�nnen, um die Effekte direkt h�rbar zu machen. Auch eine Messung des unverf�lschten Signals in einem Tester f�r integrierte Schaltkreise ist m�glich.
